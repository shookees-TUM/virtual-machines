\documentclass{report}
\usepackage[utf8]{inputenc}
\usepackage{multicol}
\usepackage{listings}
\usepackage[usenames, dvipsnames]{color}
\usepackage{fancyhdr}
\pagestyle{fancy}
\fancyhf{}
\rhead{TUM, Virtual Machines}
\lhead{Exercise 02 report}
\chead{Paulius Sukys}

\setcounter{chapter}{2}
\begin{document}
\begin{multicols}{3}
[
\section{Expression}
$x=(y=(y+4))*z$; $\rho=\{x\to4, y\to5, z\to6\}$\\
Deconstructing: $y=y + 4; x = y * z$
]
\begin{lstlisting}[
mathescape,
columns=fullflexible,
basicstyle=\fontfamily{lmvtt}\selectfont,
]
    loada 5
    loadc 4
    add
    storea 5
    loada 5
    loada 6
    mul
    storea 4
\end{lstlisting}
\end{multicols}
Since $y$ is set midway, I store the new value, which removes it from stack, then to get it back on stack, I load it back from updated $y$. Alternatively, this can be done by using duplicating:
\begin{multicols}{3}
\begin{lstlisting}[
mathescape,
columns=fullflexible,
basicstyle=\fontfamily{lmvtt}\selectfont,
]
    loada 5
    loadc 4
    add
    dup
    storea 5
    loada 6
    mul
    storea 4
\end{lstlisting}
\end{multicols}
Duplication will avoid referencing the register, although computation-wise, I do not know which solution is better.


\section{The {\fontfamily{lmvtt}\selectfont do-while} Loop}
Code generation scheme for {\fontfamily{lmvtt}\selectfont do \textit{s} while (\textit{e})}:\\
{\color{OliveGreen}$code$} $s'$ $\rho$ =\\
\hspace*{1.5cm} {\color{RoyalBlue}A:} {\color{OliveGreen}$code$} $s$ $\rho$\\
\hspace*{2cm} {\color{OliveGreen}$code_R$} $e$ $\rho$\\
\hspace*{2cm} {\color{RoyalBlue}$jumpz$ $B$}\\
\hspace*{2cm} {\color{RoyalBlue}$jump$ $A$}\\
\hspace*{1.5cm} {\color{RoyalBlue}B:} ...

\section{{\fontfamily{lmvtt}\selectfont break} \& {\fontfamily{lmvtt}\selectfont continue}
{\fontfamily{lmvtt}\selectfont break} and {\fontfamily{lmvtt}\selectfont continue}} statements are simply unconditional jumps:\\
break - outside cycle;
continue - to cycle beginning

\subsection{{\fontfamily{lmvtt}\selectfont for} loop}
{\color{OliveGreen}$code$} $s$ $\rho$ =\hspace*{.35cm} {\color{OliveGreen}$code_R$} $e_1$ $\rho$\\
\hspace*{2cm} {\color{RoyalBlue}$pop$}\\
\hspace*{1.5cm} {\color{RoyalBlue}A:} {\color{OliveGreen}$code_R$} $e_2$ $\rho$\\
\hspace*{2cm} {\color{RoyalBlue}$jumpz B$}\\
\hspace*{2cm} {\color{OliveGreen}$code$} $s'$ $\rho$\\
\hspace*{2cm} {\color{OliveGreen}$code_R$} $e_3$ $\rho$\\
\hspace*{2cm} {\color{RoyalBlue}$pop$}\\
\hspace*{2cm} {\color{RoyalBlue}$jump$ $A$}\\
\hspace*{1.5cm} {\color{RoyalBlue}B:} ...

Where {\color{RoyalBlue}$B$} is label outside cycle; {\color{RoyalBlue}$A$} - label to cycle start. Here {\color{OliveGreen}$code$} $s'$ $\rho$ would contain {\fontfamily{lmvtt}\selectfont break} and/or {\fontfamily{lmvtt}\selectfont continue} statements, where:\\
\begin{enumerate}
\item {\fontfamily{lmvtt}\selectfont break} directly translates to {\color{RoyalBlue}$jump$ $B$}
\item {\fontfamily{lmvtt}\selectfont continue} directly translates to {\color{RoyalBlue}$jump$ $A$}
\end{enumerate}
\subsection{{\fontfamily{lmvtt}\selectfont while} loop}
{\color{OliveGreen}$code$} $s$ $\rho$ =\\
\hspace*{1.5cm} {\color{RoyalBlue}A:} {\color{OliveGreen}$code_R$} $e$ $\rho$\\
\hspace*{2cm} {\color{RoyalBlue}$jumpz B$}\\
\hspace*{2cm} {\color{OliveGreen}$code$} $s'$ $\rho$\\
\hspace*{2cm} {\color{RoyalBlue}$jump$ $A$}\\
\hspace*{1.5cm} {\color{RoyalBlue}B:} ...

Where {\color{RoyalBlue}$B$} is label outside cycle; {\color{RoyalBlue}$A$} - label to cycle start. Here {\color{OliveGreen}$code$} $s'$ $\rho$ would contain {\fontfamily{lmvtt}\selectfont break} and/or {\fontfamily{lmvtt}\selectfont continue} statements, where:\\
\begin{enumerate}
\item {\fontfamily{lmvtt}\selectfont break} directly translates to {\color{RoyalBlue}$jump$ $B$}
\item {\fontfamily{lmvtt}\selectfont continue} directly translates to {\color{RoyalBlue}$jump$ $A$}
\end{enumerate}

\section{Reverse engineering}
Convert to C, given $\rho = \{ x\to1, y\to2\}$. There is an assumption done on the exercise sheet given CMa code, which has its $20^{th}$ instruction set as {\fontfamily{lmvtt}\selectfont loada 2}, which in turn, after every cycle  would reference outside given address environment ($\rho$). It should have probably been {\fontfamily{lmvtt}\selectfont loadc 2}, which references the y variable.
\begin{lstlisting}[
mathescape,
columns=fullflexible,
basicstyle=\fontfamily{lmvtt}\selectfont,
]
while (10 > x)
{
  x = x + 1;
  if (x % 2 == 0)
  {
    continue;
  }
  else
  {
    y = y + 1;
  }
}
\end{lstlisting}
EDIT: as stated, the $20^{th}$ instruction was intentional, thus the C code would be as follows:
\begin{lstlisting}[
mathescape,
columns=fullflexible,
basicstyle=\fontfamily{lmvtt}\selectfont,
]
while (x > 10)
{
  x = x + 1;
  if ((x % 2 == 0) == 0)
  {
    continue;
  }
  else
  {
    *(&(*y)) = *y + 1;
  }
}
\end{lstlisting}
The {\fontfamily{lmvtt}\selectfont *(\&(*y)) = *y + 1;} represents the logic of:
set the value of y value + 1 to value of address at y value.
\end{document}