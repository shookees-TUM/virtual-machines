\documentclass{report}
\usepackage[utf8]{inputenc}
\usepackage{multicol}
\usepackage{listings}
\usepackage[usenames, dvipsnames]{color}
\usepackage{fancyhdr}
\pagestyle{fancy}
\fancyhf{}
\rhead{TUM, Virtual Machines}
\lhead{Exercise 03 report}
\chead{Paulius Sukys}

\setcounter{chapter}{3}
\begin{document}
\section{Structs}
Address environments for $\rho_{point}$ and $\rho_{foo}$:
\begin{lstlisting}[
mathescape,
columns=fullflexible,
basicstyle=\fontfamily{lmvtt}\selectfont,
]
struct point {
    int x;
    int y;
};
struct foo {
    int a;
    struct point points[5];
    struct point *ptr;
    int b;        
} bar;
sizeof(int);// = 4
\end{lstlisting}

The address environment variables are relative to the beginning of {\fontfamily{lmvtt}\selectfont struct} address:\\
$\rho_{point} = \{x\to0, x\to4\}$\\
$\rho_{foo} = \{a\to0, points\to4, ptr\to44, b\to52\}$\\

CMa code with $\rho(bar) = 1$:\\
{\color{RoyalBlue}$code_L$} {\fontfamily{lmvtt}\selectfont (bar.ptr$\to$y)} $\rho$ = loadc 1\hspace*{1.5cm} {\small $\rho(bar)$}\\
\hspace*{3.67cm} loadc 44\hspace*{1.3cm} {\small relative address inside {\fontfamily{lmvtt}\selectfont bar}}\\
\hspace*{3.67cm} loadc 4\hspace*{1.5cm} {\small relative address inside {\fontfamily{lmvtt}\selectfont struct point}}\\
\hspace*{3.67cm} add\hspace*{2cm} {\small absolute address inside {\fontfamily{lmvtt}\selectfont bar}}\\
\hspace*{3.67cm} add\hspace*{2cm} {\small absolute address in address environment}\\

\section{Extreme pointer}
Precise upper bound for the number of stack locations that are maximally needed for their evaluation
\subsection{{\fontfamily{lmvtt}\selectfont 5 + (x * 42);}}
\begin{verbatim}
loadc 5 /*+1*/
loada x /*+1*/
loadc 42 /*+1*/
mul /*-1*/
add /*-1*/
\end{verbatim}
Having a value in the stack for $x$ address, uses additional stack space
Maximum stack size: 4 (1 for address, 3 for actions)
\subsection{{\fontfamily{lmvtt}\selectfont a[y * y + 5];}}
\begin{verbatim}
loadc a \rho /*+1*/
loada y /*+1*/
loada y /*+1*/
mul /*-1*/
loadc 5 /*+1*/
add /*-1*/
add /*-1*/
load /*0*/
\end{verbatim}
Having $a$ and $y$ as addresses, uses additional 2 slots in stack
Maximum stack size: 5 (1 for $a$, 1 for $y$ and 3 for actions)
\subsection{{\fontfamily{lmvtt}\selectfont x = y + 42;}}
\begin{verbatim}
loada y /*+1*/
loadc 42 /*+1*/
add /*-1*/
loadc x \rho /*+1*/
store /*-1*/
\end{verbatim}
Maximum stack size: 4 (1 for $y$, 1 for $x$ and 2 for actions)
\subsection{{\fontfamily{lmvtt}\selectfont while (x $<$ 42) \{ x = x + 1; a[x + 2] = 42;\}}}
\begin{verbatim}
loada x /*+1*/
loadc 42 /*+1*/
le /*-1*/
jumpz B /*-1*/
A:
loada x /*+1*/
loadc 1 /*+1*/
add /*-1*/
loadc x \rho /*+1*/
store /*-1*/
pop /*-1*/
loadc 42 /*+1*/
loadc a \rho /*+1*/
loada x /*+1*/
loadc 2 /*+1*/
add /*-1*/
add /*-1*/
store /*-1*/
pop /*-1*/
jump A
B:
\end{verbatim}
Maximum stack size: 6 (1 for $x$, 1 for $a$, 4 for actions).
This can actually be optimized by changing
\begin{verbatim}
loadc 42 /*+1*/
loadc a \rho /*+1*/
loada x /*+1*/
loadc 2 /*+1*/
add /*-1*/
add /*-1*/
\end{verbatim}
to
\begin{verbatim}
loadc 42 /*+1*/
loadc a \rho /*+1*/
add /*-1*/
loada x /*+1*/
loadc 2 /*+1*/
add /*-1*/
\end{verbatim}
thus allowing smaller maximum stack size.

\section{Function call}
\subsection{Translate to C code to CMa}
\begin{verbatim}
int f(int x, int y) {
    return x * y + 2;
}
\end{verbatim}  
\begin{multicols}{3}
\begin{lstlisting}[
mathescape,
columns=fullflexible,
basicstyle=\fontfamily{lmvtt}\selectfont,
]
  _f:
  enter q
  alloc 0
  loadr x $\rho_{f}$
  loadr y $\rho_{f}$
  mul
  loadc 2
  add
  storer -3 /* is this where we plant the return value?? */
  return
\end{lstlisting}
\end{multicols}


\subsection{Translate expression $e \equiv f(8, 5)$ into CMa}
\begin{multicols}{3}
\begin{lstlisting}[
mathescape,
columns=fullflexible,
basicstyle=\fontfamily{lmvtt}\selectfont,
]
  ...
  loadc 8
  loadc 5
  mark
  loadc _f
  call
  slide 0
  ...
\end{lstlisting}
\end{multicols}
\begin{table}[]
\hline
Mapping & Mode &  & Cyclic &  & \\ \hline
& Change from 0 to 1 &  & 5.94   & 2.64 & 1.22 \\
& Change form 1 to 0 &  & 23.66  & 10.46 & 4.84 \\ \hline
& Change form 1 to 0 &  & 8.62   & 3.60 & 1.78 \\
\end{table}

\end{document}